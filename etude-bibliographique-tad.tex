\documentclass{article}

\usepackage[utf8]{inputenc}  
\usepackage[T1]{fontenc}    

\begin{document}

\title{Sur les méthodes de résolution du Problème de transport à la demande: étude bibliographique}
\author{Adegnandjou Eddy, Walid Yahia Mahammed}
\date{13 Mars 2021}
\maketitle

\newpage

\section{Introduction}

\paragraph{}
La synthèse bibliographique effectuée dans ce rapport porte sur les méthodes de résolution du problème du transport à la demande (TAD). L'objectif de cette étude est d'aborder ce thème sous différents angles de vue, afin d'en cerner efficacement le contour et de donner une orientation future à la rédaction de notre mémoire. Nous y arrivons à travers une description synthétique des idées que suscite un questionnement systématique suivie d'une analyse critique. Ensuite nous positionnons notre thème par rapport à d'autres travaux de recherche avant de conclure pour énoncer les ressources pertinentes de la littérature scientifique utilisées.


\section{Description synthétique des idées développées}

\paragraph{}
Plus anciennement appelé Bus à la demande ou Minibus à la demande, le transport à la demande consiste à allier la flexibilité dans le choix des endroits parcourus et l’accessibilité du transport en terme de coût. Il se veut adapté à des contextes totalement hétérogènes tant en interne qu’à une échelle plus globale de géants réseaux routiers. Sa particularité est qu’il n’est activé que lorsqu’un potentiel usager en fait la demande.

\paragraph{}
D'un point de vue purement scientifique, le problème du transport à la demande se rapporte à la détermination d'un chemin ou circuit au sein d'un réseau de transport en respectant certains critères définis selon la demande et mathématiquement modélisés. Ce chemin est obtenu à l'aide d'outils algorithmiques permettant de résoudre le problème et choisis selon les objectifs visés: il peut s'agir de satisfaire des exigences de temps, de coût ou de qualités de services.

\paragraph{}
Il existe plusieurs approches de résolution des TAD. Ces méthodes peuvent être classées en trois catégories. Premièrement, on distingue les méthodes de résolution exactes qui garantissent fiabilité des résultats avec des solutions optimales. Une deuxième catégorie concerne les méthodes de résolution approchées qui ont pour objectif de résoudre le problème efficacement. Finalement, nous avons les méthodes de résolution basées sur les systèmes multi-agents qui permettent de gérer la complexité dans la résolution.

\paragraph{}
Le transport à la demande répond à une problématique d’actualité: la forte mobilité pour laquelle les moyens existants tels que le bus, le taxi et même la voiture personnelle, traditionnellement utilisés se révèlent trop contraignants pour les usagers. TAD concerne essentiellement toute personne intervenant dans un réseau routier et qui utilise un moyen de transport pour se déplacer.

\paragraph{}
Le transport à la demande bien qu'indépendant d'un contexte géographique particulier est assez développé dans certains pays d'Europe et très évolué dans les zones les plus urbanisés des États-Unis. Il se développe aussi dans d'autres régions du monde.

\paragraph{}
Le TAD est issu d'un problème plus ancien: le problème du voyageur de commerce. Ce dernier a ensuite été généralisé en problème de tournées de véhicules dont l'une des variantes est le problème du transport à la demande.

\paragraph{}
Le transport à la demande utilise les moyens de déplacement en commun comme le bus et fonctionne d'une manière assez singulière. Le système n'est déclenché que s'il y a au moins une demande d'un usager. Il utilise un réseau routier et tente de s'adapter aux préférences des utilisateurs.

\paragraph{}
Pour résoudre le problème de transport à la demande, l'approche intuitive serait d'explorer de façon exhaustive l'ensemble des solutions envisageables puis de les évaluer afin de choisir celle qui correspond au mieux à la situation. Cependant, cette démarche peut s'avérer très coûteuse, sinon irréaliste dans un délai limité. Il convient donc d'utiliser des moyens appropriés pour obtenir le chemin recherché.


\section{Analyse critique des idées, forces et faiblesses}

\paragraph{}
Les méthodes de résolution exactes permettent de mener une étude suffisamment précise et fiable. Cependant elles ont du mal à gérer la complexité des problèmes de taille considérable. Quant aux méthodes de résolution approchées, elle sont utiles pour trouver rapidement une solution admissible, même si l'optimalité de la solution trouvée n'est pas la meilleure. Enfin, les méthodes de résolution basées sur les systèmes multi-agents permettent de gérer la complexité du problème en se basant sur des algorithmes d'intelligence artificielle. Il peut être utile de se servir d'un type spécifique de méthode ou d'en combiner certaines, selon le cas.

\section{Positionnement du TAD par rapport aux autres travaux}

\paragraph{}
Le problème du transport est tout d'abord un problème d'optimisation combinatoire. Il appartient à la catégorie des problèmes d’optimisation combinatoire NP-Difficile, NP-complet et peut être mono-objectif ou multi-objectif. Il intéresse de ce fait la communauté de la recherche opérationnelle d'une part. D'autre part, les sociologues s'interrogent sur les raisons qui motivent un important flux de transport ainsi que l'impact de ce fait sur la société. Les géographes quant-à eux s'intéressent à l'interaction entre les territoires et les réseaux.


\section{Conclusion}

Le problème de transport à la demande est un problème d'actualité dont la résolution peut bénéficier des récentes avancées scientifiques d'autres domaines tels que l'intelligence artificielle. Considérant que l'étude bibliographique est un processus évolutif nécessitant une veille bibliographique, les résultats présentés dans ce rapport constituent une base de travail qui sera progressivement enrichie tout au long de la rédaction du mémoire. Ils permettront de poursuivre la synthèse bibliographique en explorant le réseau de citations, les références des publications pertinentes sur les méthodes de résolution du transport à la demande.


\section{Bibliographie}

\paragraph{}
Thierry Garaix.Étude et résolution exacte de problèmes de transport à la demande avec qualité de service. Modélisation et simulation. Université d’Avignon, 2007. Français. tel-00534894

Anas Malas. Contributions à la résolution du transport à la demande fondées sur les systèmes multi-agents. Intelligence artificielle [cs.AI]. Normandie Université, 2017. Français. NNT : 2017NORMIR07. tel-01661358v2


\end{document}